\subsection{DSSP}

Em 1983, Kabsch e Sander publicaram o algoritmo de atribuição de estruturas secundárias de proteínas que viria a ser o mais utilizado até os dias atuais, o DSSP (\textit{Dictionary of Protein Secondary Structure}). 

No trabalho, os autores afirmam que a atribuição de estruturas secundárias a partir das coordenadas atômicas de estruturas proteicas é um problema de reconhecimento de padrões. Nesse contexto, eles optaram por identificar esses padrões através de ligações de hidrogênio entre átomos da cadeia principal ao invés de ângulos $\Phi$ e $\Psi$ ou de posições relativas de $C_\alpha$. A justificativa utilizada foi que a presença ou ausência de ligações de hidrogênio poderiam ser avaliadas por um simples critério energético, enquanto que outras características precisariam do ajuste de um número maior de parâmetros. %sidenote Na época havia um pouco mais de 100 estruturas depositadas

As ligações de hidrogênio foram definidas por eles utilizando um modelo eletrostático. Nesse modelo, uma ligação de hidrogênio $HB$ ocorrerá se, e somente se, a energia $E$ for menor que -0.5 kcal/mol. Para o cálculo são utilizadas as cargas parcias $+q_1, -q_1$ nos átomos $C$ e $O$, e $-q_2, +q_2$ nos átomos $N$ e $H$, onde $q_1=0.42e$ e $q_2=0.20e$.

\begin{gather}
E < -0.5 kcal/mol \implies HB = \text{Verdade}
\intertext{onde} 
E = q_1q_2(1/r(ON)+1/r(CH)-1/r(OH)-1/r(CN))*f \label{eq:dssp_energy}
\end{gather}

Na equação \eqref{eq:dssp_energy}, $r(AB)$ é a distância interatômica entre A e B em ângstroms e o fator dimensional $f=332$. 

Os autores afirmam que, por este modelo, uma boa ligação de hidrogênio teria aproximadamente -3 kcal/mol. Assim, a escolha de um limiar em -0.5 kcal/mol torna o modelo mais tolerante à erros nas coordenadas atômicas e à ligações de hidrogênios bifurcadas \citep{Kabsch1983}.

Uma vez definido o modelo para identificar ligações de hidrogênio, essas são testadas e anotadas na cadeia polipeptídica em duas classes elementares: (1) padrão \textit{n-Turn} e (2) padrão \textit{bridge}. 

O padrão \textit{n-Turn}, onde $n \in \{3, 4, 5\}$, apresentam uma ligação de hidrogênio entre o $CO$ do resíduo $i$ e o $NH$ do resíduo $i+n$.

% # Abstract

% Atribuição de estruturas secundárias através do reconhecimento de padrões de ligações de hidrogênio e características (features) geométricas extraídas de coordenadas de X-ray.

% Estruturas secundárias cooperativas são reconhecidas por repetições  de padrões de ligações de hidrogênio. Repetições de "turns" formam hélices, repetições de bridges formam "ladders", e "ladders" conectadas formam fitas.

% # Main Ideas

% Algoritmo baseado principalmente em padrões de ligações de hidrogênio. Requer o ajuste de um menor número de parâmetros em relação aos angulos phi e psi, ou as posições dos CA.

% Padrões:

% n-turns - (onde n pode ser 3,4 ou 5) apresentam uma ligação de hidrogênio entre o CO do resíduo i e o NH do resíduo i+n.

% bridges - ligações de hidrôgenio entre resíduos distantes na sequência.

% Esses dois padrões representam todos os possíveis padrões de ligações de hidrogênio entre átomos do backbone.

% Repetições de 4-turns definem uma alpha-hélice. Repetições de bridges definem uma estrutura beta. As outras ocorrências dos padrões básicos podem representar hélices-310, helices-pi, single turns e single beta-bridges.

% # Definições





