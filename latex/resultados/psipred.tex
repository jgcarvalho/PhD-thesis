\section{Resultados de predição com o PSIPRED}

O PSIPRED é um método de predição de estruturas secundárias que utiliza redes neurais artificias em conjunto com PSSM \citep{10.1006/jmbi.1999.3091} (ver \ref{ch:rev_literatura}). O método foi originalmente treinado com informações de estrutura secundária atribuídas pelo DSSP. 

Devido as variações observadas entre os métodos de atribuição de estrutura secundária, nós realizamos a comparação dos resultados de predição com a estrutura secundária atribuída por diferentes métodos, assim como com o consenso entre os métodos de atribuição.

No conjunto de proteínas utilizado nesse trabalho, o PSIPRED comparado à atribuição pelo DSSP, demonstrou uma acurácia média (Q3) de 86\%, superior aos 78\% descrito na literatura. 

A acurácia média para a predição de fitas $\beta$, como esperado, foi inferior a predição de hélices e coils.

Um resultado interessante foi a acurácia média observada entre a predição e o consenso dos métodos de atribuição estrutura secundária. Tanto a acurácia geral (Q3) quantos a acurácia por classe (Qh, Qe, Qc) demonstrou um aumento significativo em relação a comparação individual com os métodos de atribuição. 

Isso indica que para as regiões onde não há consenso entre os métodos de atribuição, a acurácia média (Q3) é próxima ou inferior a 68\%.

$Q3_\text{consenso}*P_\text{consenso} + Q3_\text{não consenso}*P_\text{não consenso} = Q3_\text{total}$

$Q3_\text{não consenso} = \frac{Q3_\text{total} - Q3_\text{consenso}*P_\text{consenso}}{P_\text{não consenso}}$

$Q3_\text{não consenso} = \frac{0.86 - 0.92*0.75}{0.25}$

$Q3_\text{não consenso} = 0.68$

%Comentar que entre os piores resultados de acurácia há grande presença de zinc fingers resolvidas em complexo com o DNA. Discutir o sentido disso

\begin{figure}
    \includegraphics[width=\linewidth]{../figures/psipred_q3.pdf}
    \caption{}
    \label{fig:psipred_q3}
\end{figure}

\begin{figure}
    \includegraphics[width=\linewidth]{../figures/psipred_qh.pdf}
    \caption{}
    \label{fig:psipred_qh}
\end{figure}

\begin{figure}
    \includegraphics[width=\linewidth]{../figures/psipred_qe.pdf}
    \caption{}
    \label{fig:psipred_qe}
\end{figure}

\begin{figure}
    \includegraphics[width=\linewidth]{../figures/psipred_qc.pdf}
    \caption{}
    \label{fig:psipred_qc}
\end{figure}