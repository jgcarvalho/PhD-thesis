\subsection{DSSP}

# Abstract

Atribuição de estruturas secundárias através do reconhecimento de padrões de ligações de hidrogênio e características (features) geométricas extraídas de coordenadas de X-ray.

Estruturas secundárias cooperativas são reconhecidas por repetições  de padrões de ligações de hidrogênio. Repetições de "turns" formam hélices, repetições de bridges formam "ladders", e "ladders" conectadas formam fitas.

# Main Ideas

Algoritmo baseado principalmente em padrões de ligações de hidrogênio. Requer o ajuste de um menor número de parâmetros em relação aos angulos phi e psi, ou as posições dos CA.

Padrões:

n-turns - (onde n pode ser 3,4 ou 5) apresentam uma ligação de hidrogênio entre o CO do resíduo i e o NH do resíduo i+n.

bridges - ligações de hidrôgenio entre resíduos distantes na sequência.

Esses dois padrões representam todos os possíveis padrões de ligações de hidrogênio entre átomos do backbone.

Repetições de 4-turns definem uma alpha-hélice. Repetições de bridges definem uma estrutura beta. As outras ocorrências dos padrões básicos podem representar hélices-310, helices-pi, single turns e single beta-bridges.

# Definições





