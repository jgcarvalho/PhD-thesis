\subsection{Primeira geração (1957-1978)} 

Dentre os métodos de predição de estrutura secundária da primeira geração, o mais reconhecido foi desenvolvido por Chou e Fasman \citeyear{key}. 

Esse método foi desenvolvido utilizando a frequência de cada aminoácidos em cada tipo de estrutura secundária, hélices $\alpha$ ou folhas $\beta$. A análise descrita no trabalho \cite{key}, foi utilizada para determinar os parâmetros $P_\alpha$ e $P_\beta$ usados na predição.


\begin{table}[]
    \centering
    \caption{My caption}
    \label{my-label}
    \begin{tabular}{@{}llllllll@{}}
    \toprule
    \multicolumn{3}{c}{$P_\alpha$}         & \multicolumn{3}{c}{$P_\beta$}         & \multicolumn{2}{c}{$P_\alpha - P_\beta$} \\ 
    \cmidrule(r){1-3} \cmidrule(r){4-6} \cmidrule(r){7-8}
    E & 1,53 & \multirow{3}{*}{$H_\alpha$} & M & 1,67 & \multirow{3}{*}{$H_\beta$} & E                 & 1,27                 \\
    A & 1,45 &                             & V & 1,65 &                            & H                 & 0,53                 \\
    L & 1,34 &                             & I & 1,60 &                            & A                 & 0,48                 \\
    \cmidrule(r){1-3} \cmidrule(r){4-6}
    H & 1,24 & \multirow{6}{*}{$h_\alpha$} & C & 1,30 & \multirow{7}{*}{$h_\beta$} & K                 & 0,33                 \\
    M & 1,20 &                             & Y & 1,29 &                            & D                 & 0,18                 \\
    Q & 1,17 &                             & F & 1,28 &                            & L                 & 0,12                 \\
    V & 1,14 &                             & Q & 1,23 &                            & N                 & 0,08                 \\
    W & 1,14 &                             & L & 1,22 &                            & S                 & 0,07                 \\
    F & 1,12 &                             & T & 1,20 &                            & P                 & -0,03                \\
    \cmidrule(r){1-3}
    K & 1,07 & \multirow{2}{*}{$I_\alpha$} & W & 1,19 &                            & W                 & -0,05                \\
    \cmidrule(r){4-6}
    I & 1,00 &                             & A & 0,97 & $I_\beta$                  & Q                 & -0,06                \\
    \cmidrule(r){1-3} \cmidrule(r){4-6}
    D & 0,98 & \multirow{5}{*}{$i_\alpha$} & R & 0,90 & \multirow{3}{*}{$i_\beta$} & R                 & -0,11                \\
    T & 0,82 &                             & G & 0,81 &                            & F                 & -0,16                \\
    R & 0,79 &                             & D & 0,80 &                            & G                 & -0,28                \\
    \cmidrule(r){4-6}
    S & 0,79 &                             & K & 0,74 & \multirow{5}{*}{$b_\beta$} & T                 & -0,38                \\
    C & 0,77 &                             & S & 0,72 &                            & M                 & -0,47                \\
    \cmidrule(r){1-3}
    N & 0,73 & \multirow{2}{*}{$b_\alpha$} & H & 0,71 &                            & V                 & -0,51                \\
    Y & 0,61 &                             & N & 0,65 &                            & C                 & -0,53                \\
    \cmidrule(r){1-3}
    P & 0,59 & \multirow{2}{*}{$B_\alpha$} & P & 0,62 &                            & I                 & -0,6                 \\
    \cmidrule(r){4-6}
    G & 0,53 &                             & E & 0,26 & $B_\beta$                  & Y                 & -0,68                \\  
    \bottomrule
    \end{tabular}
    \end{table}

A partir dos parâmetros $P_\alpha$ e $P_\beta$ eles determinaram as seguintes regras para predição:

A. Hélices 
1. Nucleação de hélices: Regiões de seis resíduos com ao menos quatro deles sendo $h_\alpha$ ou $H_\alpha$. A formação de hélice é desfavorável se o segmento contiver um terço ou mais de hélices breakers ($b_\alpha$ ou $B_\alpha$) ou menos da metade de hélices formers.

2. Terminação de hélices: As hélices iniciadas pela regra A.1. são estendidas até a presença de um tetrapeptídeo com $\langle P_\alpha \rangle < 1.0$. Regiões de folha $\beta$ também podem terminar as regiões de hélice.

3. Prolinas não podem ocorrer na região central de hélices ou na extremidade C-terminal da mesma.

4. Bordas das hélices: Prolinas, aspartatos e glutamatos preferem a extremidade N-terminal das hélices enquanto histidinas, lisinas e argininas preferem a extremidade C-terminal.

Regra 1: qualquer segmento de seis resíduos ou mais com $\langle P_\alpha \rangle > 1.03$, $\langle P_\alpha \rangle > \langle P_\beta \rangle$ e satisfazendo as condições A.1. até A.4. é predito como hélice.

B. Folhas $\beta$
1. Nucleação: Regiões de 5 resíduos com ao menos 3 deles sendo $h_\beta$ ou $H_\beta$. A formação de fita $\beta$ é desfavorável se o segmento contiver um terço ou mais de fita breakers ($b_\beta$ ou $B_\beta$) ou menos da metade de fita formers.
2. Terminação: As fitas iniciadas pela regra B.1. são estendidas até a presença de um tetrapeptídeo com $\langle P_\beta \rangle < 1.0$. Regiões de hélice $\alpha$ também podem terminar as regiões de folhas $\beta$.
3. Glutamatos ocorrem raramente em regiões de folha $\beta$. Prolinas ocorrem raramente na região central das mesmas.
4. Bordas das folhas $\beta$: Residuos carregados ocorrem raramente na extremidade N-terminal da fita e são pouco frequentes nas regiões central e C-terminal. Triptofano ocorre com maior frequência na região N-terminal e raramente na C-terminal da fita $\beta.$

Regra 2: qualquer segmento de cinco resíduos ou mais com $\langle P_\beta \rangle > 1.05$, $\langle P_\beta \rangle > \langle P_\alpha \rangle$ e satisfazendo as condições B.1. até B.4. é predito como fita $\beta$.

Os parâmetros foram calculados utilizando 15 estruturas proteicas. A aplicação das regras a essas estruturas apresentou acurácia de 77\% na classificação dos resíduos entre hélices, fitas $\beta$ e coil. Entretanto, outros trabalhos sugerem que a acurácia do método esteja entre 50-60\%. Possivelmente, o uso do mesmo conjunto de proteínas para treinamento e teste contribuiu para que a acurácia fosse superestimada. 
