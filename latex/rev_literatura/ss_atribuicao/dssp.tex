\subsection{DSSP}

Em 1983, Kabsch e Sander publicaram o algoritmo de atribuição de estruturas secundárias de proteínas que viria a ser o mais utilizado até os dias atuais, o DSSP (\textit{Dictionary of Protein Secondary Structure}). 

No trabalho, os autores afirmam que a atribuição de estruturas secundárias a partir das coordenadas atômicas de estruturas proteicas é um problema de reconhecimento de padrões. Nesse contexto, eles optaram por identificar esses padrões através de ligações de hidrogênio entre átomos da cadeia principal ao invés de ângulos $\Phi$ e $\Psi$ ou de posições relativas de $C_\alpha$. A justificativa utilizada foi que a presença ou ausência de ligações de hidrogênio poderiam ser avaliadas por um simples critério energético, enquanto que outras características precisariam do ajuste de um número maior de parâmetros. %sidenote Na época havia um pouco mais de 100 estruturas depositadas

As ligações de hidrogênio foram definidas por eles utilizando um modelo eletrostático. Nesse modelo, uma ligação de hidrogênio $HB$ ocorrerá se, e somente se, a energia $E$ for menor que -0.5 kcal/mol. Para o cálculo são utilizadas as cargas parcias $+q_1, -q_1$ nos átomos $C$ e $O$, e $-q_2, +q_2$ nos átomos $N$ e $H$, onde $q_1=0.42e$ e $q_2=0.20e$.

\begin{gather}
E < -0.5 kcal/mol \implies HB = \text{Verdade}
\intertext{onde} 
E = q_1q_2(1/r(ON)+1/r(CH)-1/r(OH)-1/r(CN))*f \label{eq:dssp_energy}
\end{gather}

Na equação \eqref{eq:dssp_energy}, $r(AB)$ é a distância interatômica entre A e B em ângstroms e o fator dimensional $f=332$. 

Os autores afirmam que, por este modelo, uma boa ligação de hidrogênio teria aproximadamente -3 kcal/mol. Assim, a escolha de um limiar em -0.5 kcal/mol torna o modelo mais tolerante à erros nas coordenadas atômicas e à ligações de hidrogênios bifurcadas \citep{Kabsch1983}.

Uma vez definido o modelo para identificar ligações de hidrogênio, essas são testadas e anotadas na cadeia polipeptídica em duas classes, ou padrões, elementares: (1) padrão \textit{n-Turn} e (2) padrão \textit{Bridge}. 

O padrão \textit{n-Turn}, onde $n \in \{3, 4, 5\}$, apresentam uma ligação de hidrogênio entre o $CO$ do resíduo $i$ e o $NH$ do resíduo $i+n$. 

\begin{gather}
\textit{n-Turn} \impliedby Hbond(i,i+n), n \in \{3,4,5\}
\end{gather}

O padrão \textit{Bridge} pode ocorrer de duas formas, a paralela e a antiparalela.

\begin{gather}
\textit{Parallel Bridge} \impliedby [Hbond(i-1,j) \land Hbond(j,i+1)] \lor [Hbond(j-1,i) \land Hbond(i,j+1)] \\
\textit{Antiparallel Bridge} \impliedby [Hbond(i,j) \land Hbond(j,i)] \lor [Hbond(i-1,j+1) \land Hbond(j-1,i+1)]
\end{gather}

Sendo que as sequências $i-1,i,i+1$ e $j-1,j,j+1$ não apresentam sobreposição (\textit{overlap}) de resíduos entre si.

As ocorrências dos padrões elementares \textit{n-Turn} e \textit{Bridge} ao longo da cadeia polipeptídica é utilizada para a atribuição dos elementos de estrutura secundária. Como exemplos, a repetição de padrões \textit{4-Turn} consecutivos indica a ocorrência de uma hélice $\alpha$, enquanto que resíduos consecutivos que apresentem padrões \textit{Bridge} formam uma fita de uma folha $\beta$.

No trabalho, os autores mencionam que esse algoritmo proposto produz hélices mais curtas, com um resíduo a menos em cada extremidade, em relação a anotação seguindo as regras da IUPAC. Outra característica do algoritmo é que hélices que apresentem algumas ligações de hidrogênio ausentes, são mantidas como uma hélice única ao invés de multiplas hélices com \textit{kinks}. O mesmo ocorre com resíduos que formam \textit{bulges} em fitas, sendo os mesmo também anotados como parte da fita. 








